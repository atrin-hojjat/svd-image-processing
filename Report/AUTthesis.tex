%%%  کلاس AUTthesis، نسخه آبان 1397
%%%   دانشگاه صنعتی امیرکبیر                 http://www.aut.ac.ir
%%%  تالار گفتگوی پارسی‌لاتک،       http://forum.parsilatex.com
%%%   آپدیت شده در آبان 95
%%%   پشتیبانی و راهنمایی          badali_farhad@yahoo.com
%%%
%%%   بازبینی و اصلاح شده در آبان ماه 1397
%%%  Tested via TeXstudio in TeXlive 2014-2018.
%%%

%-----------------------------------------------------------------------------------------------------
%        روش اجرا.: 2 بار F1 ، 2 بار  F11(به منظور تولید مراجع) ، دوبار Ctrl+Alt+I (به منظور تولید نمایه) و دو بار F1 -------> مشاهده Pdf
%%%%%%%%%%%%%%%%%%%%%%%%%%%%%%%%%%%%%%%%%%%%%%%%%%%%%%
%   TeXstudio as your IDE
%%  برای compile در TeXstudio تنها کافی است منوی Options->Configure TeXstudio را زده و در پنجره Configure TeXstudio در بخش Build گزینه Default Compiler را به XeLaTeX تغییر دهید. سند شما به راحتی compile خواهد شد.
%   F1 & F5 : Build & view
%   F6      : Compile
%   F7      : Viewira
%   --------------
%%%%%%%%%%%%%%%%%%%%%%%%%%%%%%%%%%%%%%%%%%%%%%%%%%%%%%
%        اگر قصد نوشتن رساله دکتری را دارید، در خط زیر به جای msc،
%      کلمه phd را قرار دهید. کلیه تنظیمات لازم، به طور خودکار، اعمال می‌شود.
%%% !TEX TS-program = XeLaTeX

\documentclass[oneside,msc,12pt]{AUTthesis}
\usepackage{pgf}

\usepackage{pythonhighlight}
\usepackage{graphicx}

\usepackage{listings}
\usepackage{color}
\usepackage{enumitem}
\usepackage{amsthm}

\definecolor{dkgreen}{rgb}{0,0.6,0}
\definecolor{gray}{rgb}{0.5,0.5,0.5}
\definecolor{mauve}{rgb}{0.58,0,0.82}

\lstset{frame=tb,
  language=Python,
  aboveskip=3mm,
  belowskip=3mm,
  showstringspaces=false,
  columns=flexible,
  basicstyle={\small\ttfamily},
  numbers=none,
  numberstyle=\tiny\color{gray},
  keywordstyle=\color{blue},
  commentstyle=\color{dkgreen},
  stringstyle=\color{mauve},
  breaklines=true,
  breakatwhitespace=true,
  tabsize=3
}


%       فایل commands.tex را حتماً به دقت مطالعه کنید؛ چون دستورات مربوط به فراخوانی بسته زی‌پرشین 
%       و دیگر بسته‌ها و ... در این فایل قرار دارد و بهتر است که با نحوه استفاده از آنها آشنا شوید. توجه شود برای نسخه نهایی پایان‌نامه حتماً hyperref را 
%        غیرفعال کنید.


\input{commands}


\newcommand{\insertfig}[3]{%
\begin{figure}[htbp]%
  \centering
  \rmfamily%
  \label{#3}%
  \ttfamily%
  \resizebox{\textwidth}{!}{% Resize the figure to fit text width
    \input{#1}% Include the .pgf figure
  }%
  \rmfamily%
	\caption{#2}%
\end{figure}%
}%
\newcommand\norm[1]{\left\lVert#1\right\rVert}

\begin{document}
\baselineskip=.75cm
\linespread{1.75}
%% -!TEX root = AUTthesis.tex
%%%%%%%%%%%%%%%%%%%%%%%%%%%%%%%%%%%%
\faculty{دانشکده ریاضی و علوم کامپیوتر}
\department{گرایش ریاضی}
\fatitle{یافتن ریشه‌های چندجمله‌ای با ماتریس همراه آن}
% نام استاد(ان) راهنما را وارد کنید
% \firstsupervisor{نام کامل استاد راهنما}
%\secondsupervisor{استاد راهنمای دوم}
% نام استاد(دان) مشاور را وارد کنید. چنانچه استاد مشاور ندارید، دستور پایین را غیرفعال کنید.
% \firstadvisor{نام کامل استاد مشاور}
%\secondadvisor{استاد مشاور دوم}
% نام نویسنده را وارد کنید
\name{آترین}
% نام خانوادگی نویسنده را وارد کنید
\surname{حجت}
%%%%%%%%%%%%%%%%%%%%%%%%%%%%%%%%%%
\thesisdate{بهمن ۱۴۰۲}

% چکیده پایان‌نامه را وارد کنید
\fa-abstract{
}


% کلمات کلیدی پایان‌نامه را وارد کنید
\keywords{جبر خطی عددی}



\AUTtitle
%%%%%%%%%%%%%%%%%%%%%%%%%%%%%%%%%%
\vspace*{7cm}
\thispagestyle{empty}

\pagenumbering{alph}
\input{TOC-TOF-LOT}
\input{list-of-symbols}

\pagenumbering{arabic}
\pagestyle{style1}
%--------------------------------------------------------------------------chapters(فصل ها)
\chapter{بررسی روش \lr{SVD}}

\section{مقدمه}
روش 
\lr{SVD}
یکی از اساسی‌ترین روش‌های تجزیه ماتریس‌هاست و در حوزه‌های مختلفی مانند 
پردازیش سیگنال، آمار، یادگیری ماشین
\footnote{
  \lr{Machine Learning}
}
و \dots
کاربردهای بسیاری دارد.
در این پروژه، چند کاربرد آن در حوزه‌ی پردازش تصویر را بررسی و با روش‌های دیگر مقایسه خواهیم کرد.
به این منظور تصاویر را بصورت ماتریس‌های دوبعدی از تصاویر 
\lr{gray scale}
نشان خواهیم داد.

این روش، ماتریس را به سه ماتریس تجزیه می‌کند که این ماتریس‌ها از مقادیر ویژه‌ی 
\footnote{
  \lr{Eigen Values}
}
ماتریس اولیه به دست می‌آیند، به همین دلیل بسیاری از خواصی که این مقادیر ویژه دارند را خواهند داشت.

بطور کلی در این روش‌ها، از تجزیه
\lr{SVD}
استفاده می‌شود تا یک تقریب از ماتریس با رنک
\footnote{
  \lr{Rank}
}
پایینتر به دست بیاید.
این ماتریس می‌تواند تقریب خوبی برای 
تصویر اولیه با حجم دیتا کمتر باشد.
بعلاوه، می‌توان از مقادیر ویژه‌ی
پراهمیت تر این
ماتریس برای یافتن شباهت بین تصاویر استفاده کرد.

\section{روش \lr{SVD}}
برای یک ماتریس 
$n \times m$
دلخواه
$A$
،
تجزیه 
\lr{SVD}
این ماتریس به فرم

\begin{equation}
A = U \Sigma V^T
\end{equation}

نوشته می‌شود که در آن:

\begin{itemize}
  \item ماتریس‌های $U_{m \times m}$ و $V_{n \times n}$ متعامد اند
  \item $\Sigma_{m \times n}$ یک ماتریس قطری با عناصر نامنفی و نزولی است که به آنها \lr{Singular Values} گفته می‌شود
  \item به ستون‌های $V^T$ بردار‌های یکه راست و به ستون‌های $U$ بردار‌های یکه چپ گفته‌میشد
  \item $rank(\Sigma) = rank(A)$
\end{itemize}

در این‌تجزیه، به دلیل متعامد بودن
$U$
و
$V$
داریم 
$VV^T=V^TV=I_n$
و
$UU^T=U^TU=I_m$.
از طرفی، می‌توان با جذف کوچکترین مقادیر 
$\Sigma$
به تقریبی از 
$A$
با رنک کمتر دست یافت.




\chapter{\lr{Image Compression}}
اولین کاربرد مهم روش
\lr{SVD}
در کاهش سایز تصاویر
\footnote{
\lr{Image Compression}
}
می‌باشد.
در این بخش هدف دستیابی به تصویری با حجم کمتر و کیفیتی نزدیک به تصویر اصلی می‌باشد.
برای نمایش تصاویر، ابتدا از ماتریس‌
\lr{Gray scale}
استفاده کرده، و در آخر هم روی تصاویر رنگی این روش را تست می‌کنیم.


\section{شهود روش \lr{SVD} برای کاهش سایز}
تصویر سیاه و سفید زیر را در نظر بگیرید:

این تصویر را می‌توان با یک ماتریس
$A_{n \times m}$
نمایش داد که درایه های آن عددی بین 0(سیاه) و یک(سفید) اند.
فرض کنید تجزیه
\lr{SVD}
این ماتریس به فرم
$A=U \Sigma V^T$
باشد.
می‌دانیم
عناصر قطر
$\Sigma$
مقادیر ویژه‌ی این ماتریس
بصورت نزولی می‌باشد.
ایده‌ی اصلی
\lr{SVD}
برای کاهش سایز این ماتریس،
حذف مقادیر ویژه‌ی کم اهمیت است، به طوری که ماتریس حاصل تقریبی از همین ماتریس ولی با رنک کمتری باشد.



\section{پیاده سازی روش}
در این بخش به پیاده سازی این روش با پایتون می‌پردازیم. برای این پیاده سازی از کتابخانه های
\lr{numpy, matplotlib, Pillow}
استفاده شده.
کد مورد استفاده در این بخش در
\href{https://github.com/atrin-hojjat/svd-image-processing}{اینجا}
قابل دسترس است. برای اجرای کد یک
\lr{Virtual environment}
پایتون ساخته و
\lr{requirements.txt}
را نصب نمایید.
برای قطعه کد‌های استفاده شده
\lr{Jupyter Notebook}
آماده شده، برخی کد های نمودار ها در فایل پایتون مجزا قرار دارند.



\subsection{پیاده سازی روی تصاویر سیاه و سفید}
برای پیاده سازی این روش کاهش حجم، ابتدا از یک تصویر سیاه و سفید استفاده می‌کنیم. برای تبدیل تصویر به یک ماتریس از مقادیر سیاه و سفید از کد زیر استفاده می‌کنیم:

\begin{latin}
  \begin{python}
image_name = "cat"
image_path = "./cat.png"

from PIL import Image
import matplotlib.pyplot as plt
import numpy as np
import os

image = Image.open(image_path)

# Convert the image to grayscale (black and white)
bw_image = image.convert('L')
bw_image_array = np.array(bw_image)

plt.imshow(bw_image_array, cmap='gray')
plt.axis('off')
plt.show()

bw_image_array.shape
  \end{python}
\end{latin}

همانطور که مشاهده می‌کنید، این تصاویر را با یک ماتریس با مقادیر بین
$0$
تا
$256$
می‌توان نشان داد. در مرحله بعد، تجزیه
\lr{SVD}
این ماتریس را به دست ‌می‌اوریم


\begin{latin}
  \begin{python}
U, S, VT = np.linalg.svd(bw_image_array)
U.shape, S.shape, VT.shape
  \end{python}
\end{latin}

حال برای به دست آوردن یک تصویر کمتر، کافیست
$k$
مقدار بزرگ
$\Sigma$
را درنظر گرفته و بقیه را حذف کنیم:


\begin{latin}
  \begin{python}
compressed_image = U[:, :k] @ np.diag(S[:k]) @ Vt[:k]
  \end{python}
\end{latin}

حال می‌توانیم این تصویر را با تقریب‌های مختلف بررسی کنیم:


\begin{latin}
  \begin{python}
def compress_image(U, S, Vt, rank):
    return U[:, :rank] @ np.diag(S[:rank]) @ Vt[:rank]

def show_image(image):
    plt.imshow(image, cmap='gray')
    plt.axis('off')
    plt.show()

def save_grayscale_image(image_array, output_path):
    if image_array.dtype != np.uint8:
        image_array = image_array.astype(np.uint8)
    image = Image.fromarray(image_array, mode='L')
    image.save(output_path)
compression_ratios = [1, 4, 8, 16, 32, 64, 128, 256, 512, 750, 1000, 1024, 2048]
save_grayscale_image(bw_image_array, "cat-grayscale.png")
original_size = os.path.getsize("cat-grayscale.png")

rank_errors = []

for rank in compression_ratios:
    print("Rank", rank)
    image = compress_image(U, S, VT, rank)
    filename = f"cat-{rank}.png"
    save_grayscale_image(image, filename)
    result_size = os.path.getsize(filename)
    print(f"Size {result_size / 1024 / 1024:.2}Mb", )

    error = np.mean((image - bw_image_array) ** 2)
    print(f"Error {error}")
    rank_errors.append(error)
    show_image(image)


  \end{python}
\end{latin}

نتیجه‌ی کد با تقریب‌های مختلف را می‌توانید مشاهده‌کنید:

\begin{figure}
  \includegraphics[width=\linewidth]{../image-compression/cat-1.png}
  \caption{$rank=1$}
  \label{fig:cat-bw-rank-1}
\end{figure}

\begin{figure}
  \includegraphics[width=\linewidth]{../image-compression/cat-4.png}
  \caption{$rank=4$}
  \label{fig:cat-bw-rank-4}
\end{figure}

\begin{figure}
  \includegraphics[width=\linewidth]{../image-compression/cat-8.png}
  \caption{$rank=8$}
  \label{fig:cat-bw-rank-8}
\end{figure}

\begin{figure}
  \includegraphics[width=\linewidth]{../image-compression/cat-32.png}
  \caption{$rank=32$}
  \label{fig:cat-bw-rank-32}
\end{figure}


\begin{figure}
  \includegraphics[width=\linewidth]{../image-compression/cat-256.png}
  \caption{$rank=256$}
  \label{fig:cat-bw-rank-256}
\end{figure}


\begin{figure}
  \includegraphics[width=\linewidth]{../image-compression/cat-1024.png}
  \caption{$rank=1024$}
  \label{fig:cat-bw-rank-1024}
\end{figure}

\begin{figure}
  \includegraphics[width=\linewidth]{../image-compression/cat-2048.png}
  \caption{$rank=2048$}
  \label{fig:cat-bw-rank-2048}
\end{figure}

همانطور که مشاهده می‌کنید، حتی در $rank$
پایین، کلیتی از تصویر دیده می‌شود، و با نزدیک شدن به سایز اصلی تصویر، تفاوت تقریبا ناچیز می‌باشد.

\subsection{بررسی خطا روش}
برای بررسی خطا در این روش
از دو معیار
\lr{MSE}
و
\lr{PSNR}
استفاده می‌کنیم.
\cite{4426357}
\cite{imcprichdavad}


برای محاسبه‌
$MSE$
یا
\lr{Mean Squared Error}
کافیست تفاوت تصویر نهایی به تصویر اولیه را محاسبه، و سپس مربعات ماتریس را میانگین بگیریم:

\begin{latin}
  \begin{python}
mse = np.mean((image - bw_image_array) ** 2)
  \end{python}
\end{latin}


معیار
$PSNR$
نشان دهنده‌ی میزان نویز به دیتا در تصویر می‌باشد.
روش حاسبه‌ی
$PSNR$
به شکل زیر می‌باشد:

\begin{equation}
  PSNR = 10 \log{10} \frac{255}{MSE}
\end{equation}


\begin{latin}
  \begin{python}
    psnr = 10 * (np.log10(255 / np.sqrt(mse)))
  \end{python}
\end{latin}


حال می‌توانیم این مقادیر را برای تقریب‌های مختلف محاسبه کرده و رسم کنیم:

\begin{latin}
  \begin{python}
from PIL import Image
import matplotlib.pyplot as plt
import matplotlib
import numpy as np
import os

image_name = "cat"
image_path = "./cat.png"

image = Image.open(image_path)

# Convert the image to grayscale (black and white)
bw_image = image.convert('L')
bw_image_array = np.array(bw_image)

U, S, VT = np.linalg.svd(bw_image_array)


def compress_image(U, S, Vt, rank):
    return U[:, :rank] @ np.diag(S[:rank]) @ Vt[:rank]


compression_ratios = [1, 4, 8, 16, 32, 64, 128, 256, 512, 750, 1000, 1024, 2048]

mse_values = []
psnr_values = []

for rank in compression_ratios:
    print("Rank", rank)
    image = compress_image(U, S, VT, rank)

    mse = np.mean((image - bw_image_array) ** 2)
    psnr = 10 * (np.log10(255 / np.sqrt(mse)))
    mse_values.append(mse)
    psnr_values.append(psnr)

plt.figure(figsize=(8, 6))
plt.plot(compression_ratios, mse_values, marker='o', color='blue', linestyle='-', linewidth=2, markersize=8)
plt.title('Mean Squared Error (MSE) vs. SVD compresssion')
plt.xlabel('Rank')
plt.ylabel('MSE')
plt.grid(True)
plt.tight_layout()
plt.savefig(os.path.join("../output", "image-comp-col-mse.jpg"))
matplotlib.rcParams.update({
    "pgf.texsystem": "xelatex",
    'text.usetex': True,
    'pgf.rcfonts': False,
    "font.family": "mononoki Nerd Font Mono",
    "font.serif": [],
    #  "font.cursive": ["mononoki Nerd Font", "mononoki Nerd Font Mono"],
})
plt.savefig(os.path.join("../output", "image-comp-col-mse.pgf"))
plt.show()

plt.figure(figsize=(8, 6))
plt.semilogx(compression_ratios, mse_values, marker='o', color='blue', linestyle='-', linewidth=2, markersize=8)
plt.title('Mean Squared Error (MSE) vs. log SVD compresssion')
plt.xlabel('Rank')
plt.ylabel('MSE')
plt.grid(True)
plt.tight_layout()
plt.savefig(os.path.join("../output", "image-comp-col-mse-log.jpg"))
matplotlib.rcParams.update({
    "pgf.texsystem": "xelatex",
    'text.usetex': True,
    'pgf.rcfonts': False,
    "font.family": "mononoki Nerd Font Mono",
    "font.serif": [],
    #  "font.cursive": ["mononoki Nerd Font", "mononoki Nerd Font Mono"],
})
plt.savefig(os.path.join("../output", "image-comp-col-mse-log.pgf"))
plt.show()

plt.figure(figsize=(8, 6))
plt.plot(compression_ratios, mse_values, marker='o', color='blue', linestyle='-', linewidth=2, markersize=8)
plt.title('PSNR vs. SVD compresssion')
plt.xlabel('Rank')
plt.ylabel('PSRN')
plt.grid(True)
plt.tight_layout()
plt.savefig(os.path.join("../output", "image-comp-psnr.jpg"))
matplotlib.rcParams.update({
    "pgf.texsystem": "xelatex",
    'text.usetex': True,
    'pgf.rcfonts': False,
    "font.family": "mononoki Nerd Font Mono",
    "font.serif": [],
    #  "font.cursive": ["mononoki Nerd Font", "mononoki Nerd Font Mono"],
})
plt.savefig(os.path.join("../output", "image-comp-col-psnr.pgf"))
plt.show()


plt.figure(figsize=(8, 6))
plt.semilogx(compression_ratios, mse_values, marker='o', color='blue', linestyle='-', linewidth=2, markersize=8)
plt.title('PSNR vs. log SVD compresssion')
plt.xlabel('Rank')
plt.ylabel('PSRN')
plt.grid(True)
plt.tight_layout()
plt.savefig(os.path.join("../output", "image-comp-col-psnr-log.jpg"))
matplotlib.rcParams.update({
    "pgf.texsystem": "xelatex",
    'text.usetex': True,
    'pgf.rcfonts': False,
    "font.family": "mononoki Nerd Font Mono",
    "font.serif": [],
    #  "font.cursive": ["mononoki Nerd Font", "mononoki Nerd Font Mono"],
})
plt.savefig(os.path.join("../output", "image-comp-col-psnr-log.pgf"))
plt.show()

  \end{python}
\end{latin}


\insertfig{../output/image-comp-mse.pgf}{\lr{MSE vs Rank}}{bw_comp_mse}
\insertfig{../output/image-comp-mse-log.pgf}{\lr{MSE vs log Rank}}{bw_comp_mse_log}
\insertfig{../output/image-comp-psnr.pgf}{\lr{PSNR vs Rank}}{bw_comp_psnr}
\insertfig{../output/image-comp-psnr-log.pgf}{\lr{PSNR vs log Rank}}{bw_comp_psnr_log}

همانطور که مشاهده میشود، بخش عمده بهینه سازی تا
$rank=256$
انجام میشود که حدود
$12.5\%$
ابعاد اصلی تصویر می‌باشد.


\subsection{پیاده سازی با تصاویر رنگی}
هر تصویر رنگی، شامل سه بخش
\lr{R, G, B}
می‌باشد که هریک نمایانگر میزان یکی از رنگ‌های قرمز، سبز،و آبی اند.
برای پیاده سازی این روش در یک تصویر رنگی، کافیست برای هر یک از
\lr{channel}
ها این پروسه‌را انجام دهیم، و سپس نتیجه هر سه را کنار یکدیگر قرار دهیم:



\begin{latin}
  \begin{python}
image_name = "cat"
image_path = "./cat.png"


from PIL import Image
import matplotlib.pyplot as plt
import numpy as np
import os

image = Image.open(image_path)

# Convert the image to grayscale (black and white)
bw_image_array = np.array(image)

plt.imshow(image_array)
plt.axis('off')
plt.show()

R_U, R_S, R_VT = np.linalg.svd(image_array[:, :, 0])
G_U, G_S, G_VT = np.linalg.svd(image_array[:, :, 1])
B_U, B_S, B_VT = np.linalg.svd(image_array[:, :, 2])

compression_ratios = [1, 4, 8, 16, 32, 64, 128, 256, 512, 750, 1000, 1024, 2048]
save_grayscale_image(bw_image_array, "cat-rgb.png")
original_size = os.path.getsize("cat-rgb.png")

rank_errors = []

for rank in compression_ratios:
    print("Rank", rank)
    image_R = compress_image(R_U, R_S, R_VT, rank)
    image_G = compress_image(G_U, G_S, G_VT, rank)
    image_B = compress_image(B_U, B_S, B_VT, rank)
    image = np.dstack((image_R, image_G, image_B))

    if image.dtype != np.uint8:
        image = image.astype(np.uint8)
    filename = f"cat-{rank}.png"
    save_image(image, filename)
    result_size = os.path.getsize(filename)
    print(f"Size {result_size / 1024 / 1024:.2}Mb", )
    show_image(image)

  \end{python}
\end{latin}


نتیجه برای تصاویر رنگی به این شکل خواهد بود:


\begin{figure}
  \includegraphics[width=\linewidth]{../image-compression-color/cat-1.png}
  \caption{$rank=1$}
  \label{fig:cat-bw-rank-1}
\end{figure}

\begin{figure}
  \includegraphics[width=\linewidth]{../image-compression-color/cat-4.png}
  \caption{$rank=4$}
  \label{fig:cat-bw-rank-4}
\end{figure}

\begin{figure}
  \includegraphics[width=\linewidth]{../image-compression-color/cat-8.png}
  \caption{$rank=8$}
  \label{fig:cat-bw-rank-8}
\end{figure}

\begin{figure}
  \includegraphics[width=\linewidth]{../image-compression-color/cat-32.png}
  \caption{$rank=32$}
  \label{fig:cat-bw-rank-32}
\end{figure}


\begin{figure}
  \includegraphics[width=\linewidth]{../image-compression-color/cat-256.png}
  \caption{$rank=256$}
  \label{fig:cat-bw-rank-256}
\end{figure}


\begin{figure}
  \includegraphics[width=\linewidth]{../image-compression-color/cat-1024.png}
  \caption{$rank=1024$}
  \label{fig:cat-bw-rank-1024}
\end{figure}

\begin{figure}
  \includegraphics[width=\linewidth]{../image-compression-color/cat-2048.png}
  \caption{$rank=2048$}
  \label{fig:cat-bw-rank-2048}
\end{figure}

\insertfig{../output/image-comp-col-mse.pgf}{\lr{MSE vs Rank}}{bw_comp_col_mse}
\insertfig{../output/image-comp-col-mse-log.pgf}{\lr{MSE vs log Rank}}{bw_comp_col_mse_log}
\insertfig{../output/image-comp-col-psnr.pgf}{\lr{PSNR vs Rank}}{bw_comp_col_psnr}
\insertfig{../output/image-comp-col-psnr-log.pgf}{\lr{PSNR vs log Rank}}{bw_comp_col_psnr_log}

\chapter{نتیجه گیری}
این روش برای به دست ‌آوردن تصاویر کم حجم مفید است، اما  نتیجه آن از دید انسانی مناسب نیست.
روش‌های دیگری که تمرکز روی نزدیکی نتایج از دید انسان دارند مانند روشی که برای کاهش حجم
\lr{jpg}
استفاده می‌شود
به مراتب نتایج بهتری نسبت به این روش ارائه می‌دهند.

\section{استفاده‌های دیگر}
با این ایده‌های ابتدایی،‌می‌‌توان کاربرد‌های پیچیده‌تری از
\lr{SVD}
نیز پباده سازی کرد.

\subsection{حذف نویز}

به عنوان مثال،‌فرض کنید یک تصویر داریم که مقداری نویز دارد.
با روشی مشابه و درنظر گرفتن
$k$
مقدار اولیه مهم، می‌توان یک تقریب از تصویر به دست آورد که نویز آن تا حد خوبی کم شده
\cite{7067415}
.
البته این ساده ترین راه برای
\lr{denoising}
با
\lr{SVD}
می‌باشد و راه ‌های بهتری نیز داریم
\cite{10.1007/978-3-030-32456-8_43}
.


\begin{latin}
  \begin{python}
import numpy as np
import matplotlib.pyplot as plt
from PIL import Image

def image_to_matrix(image_path):
    img = Image.open(image_path)
    img_array = np.array(img)
    return img_array

def matrix_to_image(matrix, output_path):
    img = Image.fromarray(matrix)
    img.save(output_path)

def add_noise(image_array, noise_std=30):
    noise = np.random.normal(scale=noise_std, size=image_array.shape).astype(np.uint8)
    noisy_image = np.clip(image_array + noise, 0, 255).astype(np.uint8)
    return noisy_image

def denoise_image_svd(image_array, rank=None):
    # Apply SVD separately to each channel
    denoised_channels = []
    for channel in range(image_array.shape[2]):  # Iterate over channels (e.g., R, G, B)
        U, S, Vt = np.linalg.svd(image_array[:, :, channel], full_matrices=False)

        # Apply rank approximation
        if rank is not None:
            U = U[:, :rank]
            S = np.diag(S[:rank])
            Vt = Vt[:rank, :]

        denoised_channel = np.dot(U, np.dot(S, Vt))
        denoised_channel = np.clip(denoised_channel, 0, 255).astype(np.uint8)
        denoised_channels.append(denoised_channel)

    # Combine denoised channels into an RGB image
    denoised_image = np.stack(denoised_channels, axis=2)
    return denoised_image

# Example usage
if __name__ == '__main__':
    # Load an example image
    image_path = 'cat.png'
    original_image = image_to_matrix(image_path)

    # Add Gaussian noise to the image
    noisy_image = add_noise(original_image, noise_std=30)

    # Denoise the noisy image using SVD
    denoised_image = denoise_image_svd(noisy_image, rank=200)

    # Display and save results
    plt.figure(figsize=(10, 6))
    plt.subplot(1, 3, 1)
    plt.imshow(original_image)
    plt.title('Original Image')
    plt.axis('off')

    plt.subplot(1, 3, 2)
    plt.imshow(noisy_image)
    plt.title('Noisy Image')
    plt.axis('off')

    plt.subplot(1, 3, 3)
    plt.imshow(denoised_image)
    plt.title('Denoised Image (SVD, Rank=50)')
    plt.axis('off')

    plt.tight_layout()
    plt.show()

  \end{python}
\end{latin}

\begin{figure}
  \centering
  \resizebox{\textwidth}{!}{% Resize the figure to fit text width
    \includegraphics[width=\linewidth]{../output/denoise.jpg}
  }%
  \caption{\lr{Denoising}}
  \label{fig:img-denoise}
\end{figure}

\subsection{تشخیص چهره}
یکی دیگر از کاربرد‌های
\lr{SVD}
مقایسه تصاویر با مقایسه مقادیر اولیه آنهاست
\cite{TAI20161}
.
همچنین این مقادیر اولیه می‌توانند نقطه‌ی شروع برای یک الگوریتم یادگیری باشند
\cite{5539989}
.
a

%\include{chapter4}
%\include{chapter5}

%--------------------------------------------------------------------------appendix( مراجع و پیوست ها)
\chapterfont{\vspace*{-2em}\centering\LARGE}%

\appendix
\bibliographystyle{plain-fa}
\bibliography{references}
\chapter*{‌پیوست}
\markboth{پیوست}{}
\addcontentsline{toc}{chapter}{پیوست}
%--------------------------------------------------------------------------dictionary(واژه نامه ها)
%اگر مایل به داشتن صفحه واژه‌نامه نیستید، خط زیر را غیر فعال کنید.
%\parindent=0pt
%\include{dicfa2en}%
%\include{dicen2fa}
%--------------------------------------------------------------------------index(نمایه)
%اگر مایل به داشتن صفحه نمایه نیستید، خط زیر را غیر فعال کنید.
%\pagestyle{style7}
%\printindex
%\pagestyle{style7}
%\include{en-abstract}
%\input{en_title}
\end{document}
